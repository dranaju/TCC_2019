\section{Aprendizado de Máquina}

O aprendizado de máquina é um ramo da inteligência artificial baseada na ideia que sistemas podem aprender através de dados, identificar padrões e fazer decisões com a intervenção mínima de humanos \cite{mitchell1990machine}. Expandindo seus estudos para algoritmos e modelos estatísticos que sistemas computacionais usam para melhorar seu desempenho em tarefas específicas.

É possível classificar o aprendizado de máquina em 3 grandes categorias:
\begin{itemize}
    \item Aprendizado supervisionado: este aprendizado consiste em um sistema que possui variáveis de entrada e variáveis de saída. As variáveis de entradas são referentes a uma variável de saída determinada e o aprendizado é chamado supervisionado, pois a variável de saída, durante o treino, já é conhecida pelo algoritmo. A função desse sistema é fazer com que o algoritmo possa maximizar as predições de saída para uma resposta correta que é conhecida previamente pela rede.
    \item Aprendizado não-supervisionado: consiste em um sistema que apenas possui variáveis de entrada sem uma saída correspondente. O objetivo desse tipo de aprendizado é modelar estruturas que possam dizer mais sobre o conjunto dos dados que estão sendo estudados.
    \item Aprendizado por reforço: neste tipo de aprendizado o agente inteligente interage com um ambiente dinâmico, para desempenhar um determinado objetivo. É fornecido, ao agente, o \textit{feedback} quanto a premiações e punições, na medida em que é explorado o espaço de solução.
\end{itemize}