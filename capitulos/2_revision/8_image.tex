\section{Visão Computacional}

A visão computacional é ciência da computação e sistemas de \textit{software} que podem reconhecer e entender uma imagem ou cena \cite{forsyth2002computer}.
A visão computacional também é composta de vários aspectos tais como reconhecimento de imagem, detecção de objetos geração de imagem, super-resolução de imagem e mais.
Do ponto de vista da engenharia, ela procura automatizar tarefas que o sistema visual humano pode fazer.
A detecção de objetos é provavelmente o mais profundo aspecto da visão computacional devido ao grande número de casos práticos.

\subsection{Rastreamento de Objetos}

Para fazer o rastreamento de objetos primeiramente é preciso detectar um objeto.
A detecção de objetos se refere a capacidade de computadores e sistemas de \textit{software} de localizar em um imagem ou cena e de identificar cada objeto.
Sendo amplamente utilizado para detecção de faces, detecção de veículos, pedestres e carros autônomos.
Existem muitas maneiras que a detecção de objetos pode ser usada. Em muitos casos a cor do objeto é usado para sua detecção.

O rastreamento de objetos é o processo de localizar um objeto alvo movendo em quadros consecutivos de um vídeo \cite{bascle1995region}. Essa associação pode ser especialmente difícil quando o objeto se move rápido em relação a taxa de quadros do vídeo. Outra situação que aumenta a complexidade do problema é quando o objeto rastreado se move em relação ao tempo. Para essas situações geralmente são aplicados modelos de movimento em que é descrito como a imagem do alvo pode mudar para diferente movimentos dos objetos. Essas aplicações se encontram muito relacionadas com a robótica. Já que muitas aplicações usam de visão para poder completar suas tarefas.